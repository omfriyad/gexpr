
\documentclass[12pt,a4paper] {article}

	\author
	{	
		\\ \\ \\ \\ 
		Submitted by \\ \\
		Omar Faruk Riyad \\ 
		Arif Ur Rahaman Chowdhury Suhan
	}
	\title
	{
		CSE468 Report
	}

	\date{
		Duration : Last whole week
	}



\begin{document}

	\maketitle
	\pagebreak


	%Overview of your work and results
	\section{Introduction:}
	Human visualization is a sweet gift from The Almighty Creator.And we know it can't be perfectly replicated 
	into a thing so that a machine can see.But the blind saw dream and we got hope.After the birth of Machine Learning, 
	Computer Vision started a new era.
	\\ \\
	In this report, we basically focused on the feature extraction rather than developing a classification model.
	various kind of features like Color Histogram,Histogram of Gradient andScale Invariant Feature Transform
	
	
	%Detailed, wherever possible mathematical, description of 1-NN, Metrics, Features.
	\section{Literature Review:}

	\subsection{1-NN:}
	It is a simple algorithm for single classification.
	Mathematically, a matrix is represented as:


	\subsection{Metrics:}
	A matrix is an entity composed of components arranged in rows and columns. 


	\subsection{Features:}
	A Feature describes charactristics or information present in an image.

	Most algorithms in Computer Vision strive to identify features that can be used, for instance, for recognition or classification. A feature can also be a particular combination of simpler features, with some associated metrics and geometries. For instance, many face recognition algorithms are based on locating the main features of a face (eyes, nose, mouth, etc.) in particular dispositions and relative proportions; a face thus recognized becomes a feature itself (with associated attributes) that can be used to perform similarity searches for the face in question in a database.
	

	%Description of the data set, how it was partitioned, how the comparison of the performance of the different features were done and what metrics were calculated.
	\section{Experimental setup:}


	%Presentation of the results in a sensible and readable format. Just presentation is not enough. You will need to analyze the results, especially failure cases on a per-feature basis and provide your justified insights why things work and why things do not work.
	\section{Resutls and discussion:}


	%The conclusion of your report
	\section{Conclusion:}



\end{document}


