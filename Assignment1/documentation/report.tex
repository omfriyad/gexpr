
\documentclass[12pt,a4paper] {article}
\usepackage{amsmath}

	\author
	{	
		\\ \\ \\ \\ 
		Submitted by \\ \\
		Omar Faruk Riyad \\ 
		Arif Ur Rahaman Chowdhury Suhan
	}
	\title
	{
		CSE468 Report
	}

	\date{
		Duration : Last whole week
	}



\begin{document}

	\maketitle
	\pagebreak


	%Overview of your work and results
	\section{Introduction:}
	Human visualization is a sweet gift from The Almighty Creator.And it can't be perfectly replicated 
	into a thing so that a machine can see.But the blind saw dream and we got hope.After the birth of Machine Learning, 
	Computer Vision started a new era.
	\\ \\
	The report is focused on the feature extraction rather than developing \\ a classification model.Using 
	various kind of features like Color Histogram, \\ Histogram of Gradient and Scale Invariant Feature Transform, 
	some collected datasets(images) are classified.
	
	
	%Detailed, wherever possible mathematical, description of 1-NN, Metrics, Features.
	\section{Literature Review:}

	\subsection{1-NN:}
	The 1 Nearest Neighbor classifier is the simple and one of the oldest methods known.It's 
	estimation is often low but the variance is high.The 1NN classifier idea is 
	extremely simple: to classify \textbf{X} find its closest neighbor among the training
	points (call it \textbf{Y}) and assign to \textbf{X} the label of \textbf{Y}. 
	\\ \\
	The good thing about 1NN is that it is conceptually simple and does not require learning model.Only 
	few dataset is enough for it. Also it works for low dimensional complex decision surfaces.But 
	for fixed,\textbf{K} it is asymptotically suboptimal,not by much.It's classification is slow and 
	suffers a lot from the curse of dimensionality.

	\subsection{Metrics:}
	A matrics are 2D form of an entity composed of components arranged in rows and columns. 
	\begin{gather}
		M = 
 		\begin{bmatrix} 
 			m_{00} & m_{01} & m_{02}\\ 
 			m_{10} & m_{11} & m_{12}
 		\end{bmatrix}
	\end{gather}
	Here M matrix contains 2 rows and 3 columns. \\
	In computer vision, there are many types of matrics like: 
	\begin{itemize}
	  \item Fundamental matrix: A 3x3 matrix which relates corresponding points in stereo images in pixel coordinates.
	  \item Essential matrix: A 3x3 matrix that also relates corresponding points but normalized pixel coordinates.
	  \item Camera matrix: A 3x4 matrix describes the mapping of a pinhole camera from 3D points in the world to 2D points in an image.
	\end{itemize}


	\subsection{Features:}
	A feature is a 1D vector that aims to represent an object or a part of an object in a compact.
	This means that features for different objects or object parts must be different and just by looking at the feature, 
	one must be able to tell what object that is. That feature not change for different appearances of the object, 
	which may be due to lighting, shadows, pose, etc. Also, one does a lot of computations with features, 
	which is why it's size must be small.

	Most algorithms in Computer Vision strive to identify features that can be used, for instance, for recognition or classification. 
	A feature can also be a particular combination of simpler features, with some associated metrics and geometries.
	They can be used for : 
	\begin{itemize}
		\item Interest point detection
		\item Descriptors and
		\item Matching of an image
	\end{itemize}
	

	%Description of the data set, how it was partitioned, how the comparison of the performance of the different features were done and what metrics were calculated.
	\section{Experimental setup:}




	%Presentation of the results in a sensible and readable format. Just presentation is not enough. You will need to analyze the results, especially failure cases on a per-feature basis and provide your justified insights why things work and why things do not work.
	\section{Results and discussion:}


	%The conclusion of your report
	\section{Conclusion:}

\end{document}


